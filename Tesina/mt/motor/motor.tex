\subsection{Motor trifásico a inducción}

\subsubsection{Introducción}

El motor trifásico de inducción es una máquina eléctrica rotativa que convierte energía eléctrica en energía mecánica. Es el tipo de motor más utilizado en la industria debido a su robustez, bajo mantenimiento y buena relación costo-rendimiento. Su funcionamiento se basa en el principio de la inducción electromagnética, según el cual una corriente eléctrica puede generarse en un conductor expuesto a un campo magnético variable.
El motor está compuesto principalmente por el estator y el rotor.

El estator es la parte fija y contiene los devanados trifásicos alimentados por una corriente alterna trifásica. El rotor, en cambio, es la parte móvil que gira dentro del estator y suele ser de tipo jaula de ardilla, formado por barras conductoras cortocircuitadas en sus extremos mediante anillos.

\subsubsection{Principio de funcionamiento}

Cuando el estator recibe una tensión trifásica equilibrada, en cada uno de sus devanados circula una corriente alterna desfasada 120° eléctricos respecto de las demás. Como resultado, cada fase genera un campo magnético alterno que, al combinarse con los otros dos, produce un campo magnético giratorio que mantiene su magnitud constante y rota a una velocidad síncrona, determinada por la frecuencia de alimentación y el número de polos del motor.
\[
n_s=\frac{120 \times f}{p}
\]
Donde $n_s$ es la velocidad síncrona en revoluciones por minuto (rpm), $f$ es la frecuencia de la red y $p$ es el numero de polos del motor. 

\paragraph{Campo magnético giratorio y generación del par}

El campo magnético giratorio del estator corta los conductores del rotor, generando una fuerza electromotriz inducida (f.e.m.) debido al movimiento relativo entre el campo y el rotor. Como las barras del rotor forman un circuito cerrado, se generan corrientes inducidas que interactúan con el campo del estator, produciendo fuerzas electromagnéticas que originan el par motor y provocan el giro del rotor en el mismo sentido que el campo giratorio.

El rotor nunca alcanza exactamente la velocidad síncrona ya que si lo hiciera no existiría movimiento relativo ni inducción. La diferencia entre la velocidad síncrona y la velocidad del rotor se denomina deslizamiento, y su valor depende de la carga mecánica aplicada.

\subsubsection{Control mediante inversor (VFD)}

En los sistemas modernos, la velocidad de giro del motor se controla mediante un inversor de frecuencia (VFD). Este dispositivo ajusta la frecuencia y la tensión aplicadas al motor, modificando así la velocidad del campo magnético giratorio del estator y, en consecuencia, la velocidad del rotor.

Sin embargo, al cambiar la frecuencia también se ve afectada la reactancia inductiva de los devanados del estator, que está dada por $X_L=2\pi \ f \ L$. Si la frecuencia disminuye y la tensión se mantiene constante, la corriente magnetizante aumenta excesivamente, saturando el núcleo del motor y provocando sobrecorrientes. Por el contrario, si la frecuencia aumenta sin ajustar la tensión, el flujo magnético disminuye y el motor pierde par.

Por esta razón, el inversor varía simultáneamente la tensión y la frecuencia manteniendo aproximadamente constante la relación $V/f$, de modo que el flujo magnético del motor se mantenga constante. Esto permite que el motor desarrolle un par nominal constante en todo el rango de velocidades, asegurando un funcionamiento estable, eficiente y sin sobrecalentamiento.

