\label{sec:3_mod}

\subsection{Modulación}

La implementacion de la modulacion se realiza en el STM32 el cual tiene una performance suficiente para manejar nuestra modulacion, ademas de los perisfericos adecuados.

La modulacion SVM necesito una salida PWM alineada al centro. Esto se logra configurando el timer 3 del microcontrolador en modo 3 en donde el contador asciende, desciende y hace comparaciones en ambos caminos. En este nos permite incluir tres valores para comparar, los cuales estan asociados a $t_1$, $t_2$, $t_0$ y uno extra para recargar los valores. En estas comparaciones se genera una interrupcion interna en donde se conmutan los pines necesarios.

La frecuencia de salida de este timer 3 esta seteada para que sea de $2.5kHz$.

En paralelo, el timer 2 que se dispara a una frecuencia mayor y es el encargado de realizar los calculos de los siguientes valores de conmutacion. Para reducir los tiempos de calculos se realiza una aproximacion en donde el error cometido es bajo. Esto se debe a que las expresiones de calculo tienen funciones trigonometricas como seno y coseno.
Partiendo de las ecuaciones de los tiempos antes descriptas, se calcula para incrementos de 1 grado los valores de tiempo y este se transforma en ticks necesarios del microcontrolador para el siguiente cambio. Con estos valores se aplica una regresión lineal y se encuentran los valores del termino lineal e independiente.

\[
tick_1 = M_1\left[ A_1 * \alpha + B_1 \right]
\]
\[
tick_2 = M_2\left[ A_2 * \alpha + B_2 \right]
\]
\[
tick_0 = 255 - tick_1 - tick_2
\]

\begin{figure}[H]
\centering
    \includegraphics[width=0.7\textwidth]{dev/mod/img/RegresTicksMod.png}
    \caption{Curvas y aproximaciones de los tiempos de conmutación $t_1$, $t_2$ y $t_0$.}
\label{fig:imgDev_Mod_RegresTicksMod}
\end{figure}

Los valores de angulos y velocidades se utilizaron escalados en 1e6 para permitir multiplicaciones usando números enteros y que los truncamientos no generen errores apreciables. 