\label{sec:3_mod}

\subsection{Modulación}

La implementación de la modulación se realizó en un microcontrolador STM32, cuya capacidad de procesamiento y conjunto de periféricos resultan adecuados para ejecutar el esquema de modulación requerido.

\paragraph{Timer 3} La modulación SVM exige la generación de señales PWM centradas. Para ello se configuró el Timer 3 de manera tal que el contador ascienda y descienda, efectuando comparaciones en ambos sentidos para que, además, permita definir tres valores de comparación asociados a $t_1$, $t_2$, $t_0$ y un valor adicional para la recarga. Cada comparación genera una interrupción interna mediante la cual se actualizan los pines de conmutación correspondientes. La frecuencia de operación del Timer 3 se fijó en $2.5kHz$.

\paragraph{Timer 2} En paralelo, el Timer 2 opera a una frecuencia mayor y es el encargado de calcular los valores de conmutación para el siguiente ciclo. Con el fin de reducir el tiempo de cómputo, se aplicó una aproximación que introduce un error despreciable respecto de las expresiones exactas, las cuales incluyen funciones trigonométricas (seno y coseno) donde el error introducido puede observarse en la figura~\ref{fig:imgDev_Mod_RegresTicksMod}. A partir de las ecuaciones de los tiempos previamente definidas, se calcularon los valores correspondientes a incrementos de $1$°, convirtiéndolos luego a la cantidad de ticks del microcontrolador necesarios para la actualización siguiente. Sobre estos datos se aplicó una regresión lineal, obteniéndose los coeficientes lineales e independientes utilizados en tiempo real

\[
tick_1 = M \left[ A_1 * \alpha + B_1 \right]
\]
\[
tick_2 = M \left[ A_2 * \alpha + B_2 \right]
\]
\[
tick_0 = 255 - tick_1 - tick_2
\]

\begin{figure}[H]
\centering
    \includegraphics[width=0.7\textwidth]{dev/mod/img/RegresTicksMod.png}
    \caption{Curvas y aproximaciones de los tiempos de conmutación $t_1$, $t_2$ y $t_0$.}
\label{fig:imgDev_Mod_RegresTicksMod}
\end{figure}

Los valores de ángulos y velocidades se escalaron por un factor de $10^6$, permitiendo realizar todas las operaciones utilizando enteros y minimizando errores significativos por truncamiento.