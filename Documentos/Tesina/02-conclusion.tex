\label{sec:5_concl}

A partir del desarrollo realizado, puede concluirse que el proyecto cumple con los objetivos propuestos para un variador de frecuencia trifásico alimentado desde baja tensión orientado a aplicaciones de potencia moderada. El sistema diseñado permite el control adecuado de un motor trifásico mediante modulación vectorial \textit{SVM}, logrando una generación de tensiones adecuadas, un buen aprovechamiento del bus de continua y un comportamiento estable en régimen permanente.

Desde el punto de vista energético y térmico, el diseño demuestra una capacidad de sobrecarga limitada pero controlada: el variador es capaz de sostener una potencia del orden de 1/3 HP durante intervalos de tiempo reducidos, lo cual resulta adecuado para condiciones transitorias como arranques, picos de par o demandas momentáneas. Sin embargo, para operación continua, el sistema mantiene de forma confiable una potencia de 1/6 HP, nivel en el cual se garantiza el cumplimiento de las restricciones térmicas, eléctricas y de confiabilidad de los componentes utilizados.

Asimismo, la arquitectura adoptada, tanto en la etapa conversora como en la etapa inversora, junto con los programas de modulación y la HMI desarrollada, permite un funcionamiento robusto y repetible, acorde con un prototipo funcional. En este sentido, el proyecto establece una base sólida para futuras iteraciones orientadas a incrementar la potencia continua, mejorar la precisión en las mediciones de tensión y corriente del bus de continua, e implementar estrategias de control más avanzadas, como el control de par.

En conclusión, el variador desarrollado cumple con las características establecidas en el alcance del proyecto, satisfaciendo los requisitos funcionales y operativos definidos, y evidenciando un equilibrio adecuado entre simplicidad de implementación, nivel de desempeño alcanzado y las limitaciones inherentes a un sistema de baja potencia y bajo costo.