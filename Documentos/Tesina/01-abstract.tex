\label{sec:1_abstract}

Este trabajo presenta el diseño, implementación y validación de un controlador de motor trifásico para instalaciones de baja tensión, compuesto por un convertidor DC-DC aislado y un inversor trifásico gobernado por modulación de vectores espaciales (SVM). El convertidor, de topología push-pull, eleva una entrada de 12 Vdc a un bus de continua de 311 Vdc, optimizando el uso del transformador de ferrita y los semiconductores. El inversor sintetiza tensiones trifásicas ajustables en frecuencia y amplitud para accionar un motor de inducción, manteniendo la relación V/f y priorizando la eficiencia con conmutación ordenada. El sistema incorpora una HMI basada en ESP32 con interfaz local (display SH1106, teclado matricial, expansión de GPIO con MCP23017) y configuración vía Wi-Fi; la lógica de control en STM32 se coordina mediante un enlace SPI maestro-esclavo. Se desarrollaron PCB dedicadas para potencia y control, con aislamiento galvánico y protecciones. Los ensayos funcionales abarcan integridad de la modulación, simetría de pulsos, índice de modulación y pruebas de rendimiento eléctrico y térmico, confirmando la viabilidad del enfoque para aplicaciones de bajo costo, mantenimiento simple y operación segura en ambientes de potencia moderada. El documento detalla criterios de diseño magnético, consideraciones constructivas, arquitectura de firmware y resultados de laboratorio.

%\keywords{ variador de frecuencia, SVM, push-pull, inversor trifásico, motor de inducción, ESP32, STM32, HMI,conversión DC-DC. }

\textbf{Palabras clave:} variador de frecuencia, SVM, push-pull, inversor trifásico, motor de inducción, ESP32, STM32, HMI,conversión DC-DC.