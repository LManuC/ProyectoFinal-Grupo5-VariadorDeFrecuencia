
\subsection{Cálculo de transformador}
Para que el variador pueda entregar una potencia continua del orden de $240W$, resulta necesario un rediseño del transformador de la etapa conversora. El núcleo utilizado en el prototipo presenta limitaciones térmicas que impiden su operación sostenida a dicha potencia, por lo que se propone el uso de un núcleo de mayores dimensiones. En particular, se seleccionó el modelo $ETD49/25/16\text{-}3C94$ del fabricante $Ferroxcube$, cuyas características lo hacen adecuado para aplicaciones de conversión de potencia en el rango considerado. A continuación, se detallan sus principales parámetros.

\begin{table}[H]
\centering
\renewcommand{\arraystretch}{1.2}
\begin{tabular}{|l|l|}
\hline
\textbf{Parámetro} & \textbf{Valor} \\
\hline
Aleación & MgZn \\
Forma & ETD49 \\
Material & N87 \\
Color & Negro \\
\hline
\end{tabular}
\caption{Información general del núcleo ETD49}
\end{table}


\begin{table}[H]
\centering
\renewcommand{\arraystretch}{1.2}
\begin{tabular}{|l|l|}
\hline
\textbf{Parámetro} & \textbf{Valor} \\
\hline
$B_{sat}$ & 320 mT \\
$\mu_i$ (25°C) & 1630 \\
\hline
\end{tabular}
\caption{Características magnéticas del núcleo ETD49}
\end{table}


\begin{table}[H]
\centering
\renewcommand{\arraystretch}{1.2}
\begin{tabular}{|l|l|}
\hline
\textbf{Parámetro} & \textbf{Valor} \\
\hline
Largo efectivo $l_e$ & 114 mm \\
Área transversal $A_c$ & 211 mm$^2$ \\
Área ventana $A_w$ & 343 mm$^2$ \\
Volumen efectivo $V_e$ & 24100 mm$^3$ \\
Dimensión A & 48.5 mm \\
Dimensión B & 37.9 mm \\
Dimensión C & 24.9 mm \\
Dimensión D & 17.7 mm \\
Dimensión E & 16.7 mm \\
Dimensión F & 16.7 mm \\
\hline
\end{tabular}
\caption{Parámetros geométricos y efectivos del núcleo ETD49}
\end{table}


\begin{figure}[h!tbp]
\centering
    \includegraphics[width=0.9\textwidth]{mt/cnv/img/EsqDimEDT42.png}
    \caption{Referencias de medidas del transformador.}
\label{fig:imgMTEsqDimEDT42}
\end{figure}


\paragraph{Relación de transformación y cálculo de espiras}

Para el cálculo del número de espiras se adoptó una relación de transformación de $1{:}40$, incorporando un factor de sobre-relación similar al utilizado en el prototipo, aunque levemente inferior. Esta relación equivale a un factor aproximado de $1.5$, permitiendo alcanzar la tensión de bus requerida con un margen adecuado de regulación.

El devanado primario se diseñó con una configuración de $3 + 3$ espiras con toma central, mientras que el devanado secundario está compuesto por un total de $120$ espiras.

\paragraph{Sección de conductores}

En el dimensionamiento de los conductores se mantuvo una densidad de corriente de $3A/mm^2$ para el primario, mientras que en el secundario se adoptó un valor de $5A/mm^2$, acorde a la menor corriente circulante.

Suponiendo una forma de onda cuadrada con un ciclo de trabajo del $45\%$, la corriente eficaz del primario resulta de $13.8,A$. A partir de este valor se determinaron secciones mínimas de $4.6,mm^2$ para el primario y $0.3,mm^2$ para el secundario. Finalmente, se seleccionaron conductores de $5,mm^2$ para el primario y un doble alambre de $0.4,mm$ de diámetro para el secundario, verificándose que la ocupación de la ventana del núcleo se mantuviera dentro de valores admisibles.